\begin{chapterquote}
   ``(..) I realized that actually doing physics is much more enjoyable than just
   learning it. Maybe 'doing it' is the right way of learning, at least as far as
   I am concerned. ''
  \quoteauthor{Gerd Binnig, Nobel Prize Laureate in Physics, 1986}
\end{chapterquote}

\chapter{Conclusions}

\section{Numerical considerations}

A multi-species, multi-dimensional, steady-state, and time-accurate numerical method
is developed to solve the Favre-averaged Navier-Stokes equations closed by the
Wilcox $k\omega$ model. The flux discretization is based on the Yee-Roe scheme
and the pseudotime stepping on block-implicit approximate factorization.
The Wilcox $k\omega$ model along with the Wilcox dilatational dissipation
are deemed adequate in capturing the essentials of the
flowfield physics and good agreement is shown with the
experimental data of a ramp injector by Waitz {\it et al.} \cite{aiaa:1993:waitz}
and a swept ramp injector by Donohue {\it et al.} \cite{aiaa:1994:donohue}.
For a typical cantilevered ramp
injector flowfield over a flat plate, a grid refinement study is performed
and a relative error of approximately 10--15\% on the mixing efficiency is estimated
for the baseline mesh (using a grid dimensions factor $r=1$). The grid-induced
error for the three-dimensional cantilevered ramp injector flowfields is here estimated
by comparing the effect of the grid to similar two-dimensional
mixing problems where the grid can be refined sufficiently. \emph{The Richardson extrapolation
is here observed to be inadequate in determining the grid-induced error of
a mixing layer problem, due to the discontinuous turbulent/non-turbulent interface.}
The influence of the turbulent Schmidt number on the mixing
efficiency is seen to be minimal for a hydrogen-air mixing problem, as the mixing
efficiency increases by only 8\% for a turbulent Schmidt number variation from 1.0 to 0.25.
The use of the Yee entropy correction factor in conjunction with the Roe scheme
is found to be unnecessary for the inlet flowfields and its use
should be avoided as it increases significantly the numerical error in the mixing layer.


\section{Marching window acceleration technique}

Dubbed the marching window, a novel acceleration technique is presented which
is aimed at accelerating the convergence of the Favre-averaged Navier-Stokes
equations in the supersonic/hypersonic regime for flowfields with large streamwise
separated flow regions. Similarly to the active domain method \cite{aiaa:1997:nakahashi},
the marching window iterates in pseudotime a band-like computational domain of minimal
width which adjusts to the size of the streamwise elliptic regions when encountered.
However, in contrast to the active domain method, it is shown that \emph{the marching window
guarantees the residual on all nodes to be below a user-defined convergence
threshold when convergence
is reached}, and hence results in the same
converged solution (within the tolerance of the convergence criterion) as the
one obtained by standard pseudotime marching methods. Further,
a streamwise ellipticity sensor based on the Vigneron splitting
\cite{aiaaconf:1978:vigneron} is developed which ensures
the downstream boundary of the marching window to advance sufficiently
such that regions of significant streamwise
ellipticity are contained within the marching window subdomain.
It is noted that while the Vigneron splitting sensor does not capture all possible
streamwise elliptic phenomena, this does not affect the accuracy of the final solution
and only affects the performance of the marching window as an acceleration
technique. Also, a multizone decomposition is implemented inside the marching window
to restrict the computing to the zones where the residual is above the user-defined
threshold. This is shown to further decrease the work needed for convergence by close to
2 times for the test cases shown in Chapter \ref{chapter:numerical_method}.


The use of the marching window with multizone decomposition
on a backward facing step and a shock boundary
layer interaction flowfield (where one or several large streamwise separated region
is present) reveals a 4 to 6 times decrease in storage and a 4 to 8 times
decrease in work compared to the standard cycle.
The proposed algorithm is also shown to work well at
a low CFL number in regions of quasihyperbolicity/parabolicity
and is recommended for stiff
problems with high non-linear stability restrictions on the time step size.
For the inlet cases presented in Chapter ..., the marching window
with multizone decomposition results in convergence attained in typically less than
200 effective iterations using a CFL number of 0.6.
A variant of the marching window designed for time-accurate simulations
is observed to result in a fivefold reduction in storage and 25\% reduction in work
for the time-accurate exploding cavity case investigated in Chapter
\ref{chapter:numerical_method}.
The reduction in computational work through the use of the marching window
is made possible by focusing the
high number of iterations needed to converge the streamwise
separated regions to the region in question, and not to the
entire computational domain. The amount of storage needed is also significantly
reduced if no memory is allocated to the nodes outside of the marching window subdomain.


\emph{The marching window does not impose any restriction
on the discretization stencils part of the residual or on the pseudotime
stepping method.}  While not implemented here,
the numerous acceleration techniques available for pseudotime
stepping (such as multigrid, block relaxation \cite{aiaa:2000:denicola},
preconditioning, Newton-Krylov, \etc) can be used in conjunction with
the marching window.
Furthermore, the marching window is not limited to the Favre-averaged
Navier-Stokes equations and its application to other fields of physics
where a quasihyperbolicity of the system is present would only
require a redefinition of the ellipticity sensor shown in Eq.~(...).


The ellipticity threshold constant $\varphiverge$ and the marching window
minimal width $\phi_3$ are seen to affect the performance of the algorithm
significantly, and it is unclear at this stage by how much these
parameters would need to be altered for very different flow properties and physical domain size.
The dependency on the problem setup seems not too severe as the same
values for the user-specified constants are used for all inlet cases, with a resulting
similar convergence rate.



\section{Thrust potential}

The losses and the gains in the flowfields presented herein are assessed
mostly by the air-based mixing efficiency and by the thrust potential. It is
recalled that the thrust potential at a certain station corresponds
to the thrust of the vehicle obtained if the flow is reversibly expanded
from the station in consideration to either the engine exit area or the engine
back pressure.
In Chapter ..., it is shown that the thrust potential
can be expressed as a function
of the stagnation pressure, stagnation temperature, and backpressure
of the engine. While limited to a perfect gas, Eq.~(...)
shows that losses or gains in an engine cannot be determined solely by monitoring the
stagnation pressure.
This is confirmed by the inlet cases presented herein: at the station where
fuel is injected, the thrust potential increases significantly due to
the high momentum of the fuel injected while the mass flux averaged stagnation pressure
decreases. Another conclusion drawn from Eq.~(...) is that the
thrust potential can be expressed solely as a function of the stagnation temperature
and the ratio between the stagnation pressure and the backpressure. Hence, it follows
that \emph{variations
in the backpressure impact the thrust potential to the same degree as variations
in stagnation pressure}.
In a scramjet or a shcramjet engine, the flow is typically underexpanded, and
the backpressure of the engine is known to be significantly higher than the surroundings.
Expanding the flow to a fixed backpressure equal to the surroundings would hence
lead to significant errors in the thrust potential.
Therefore, the thrust potential is here determined by reversibly expanding the
flow to an iteratively determined backpressure, which is such that the cross-sectional
area of the expanded flow corresponds to the engine exit area. It is noted
that the same backpressure is shared by all streamtubes at one particular
station. In this way, we avoid errors originating
from a fixed backpressure and errors related to unrealistic
backpressure variations that would occur when expanding the flow to a constant area.
For a typical inlet flowfield, the backpressure is seen to increase by approximately
2.5 times while the mass-flux averaged stagnation pressure decreases by 4 times.
\emph{The similar observed changes in the backpressure and the stagnation
pressure shows the importance of accurately determining the backpressure when assessing
the losses in a shcramjet inlet}.

\section{Fundamental mixing studies}

%mention the conclusions from the parametric study of mixing enhancement by compression
To predict the mixing efficiency increase through a compression wave, an expression
is derived for the special case of hypersonic flow at a high convective Mach number.
In such conditions, it is seen that two assumptions can be made: (i)
the speed of the fuel and air streams
remain constant through the compression and (ii) the density increase of the air
corresponds to the density increase of the fuel through the compression. From these
two assumptions, it is then shown in the mixing efficiency growth
equation [\ie\ Eq.~(...)]
that \emph{the mixing efficiency growth is proportional to the product
of the flow density, the interface length, and the Papamoschou-Roshko correction
term.} Noting that the Papamoschou-Roshko correction term decreases for decreasing
temperature, this shows one of the major challenges of mixing in the inlet.
\emph{The very low flow density and flow temperature
lead to a very low mixing
efficiency growth in a shcramjet inlet, as compared to the mixing efficiency growth
that could be obtained in the combustor, where the temperature and the density are high.}
With the help of the derived mixing efficiency growth,
the separate effect of interface stretching due to vorticity is assessed
for a mixing layer traversing an oblique shock and a mixing layer traversing
a Prandtl-Meyer compression fan. It is observed that \emph{the higher density induced
by the compression fan leads to a greater increase in the mixing efficiency growth,
despite the more vigorous interface stretching by the axial vortices induced
by the oblique shock.}


% fundamental study of importance of ramp-induced axial vortices
The increase in mixing efficiency generally associated to ramp-like injector
configurations compared to planar mixing configurations is also analyzed.
A comparison is performed between the mixing efficiency
obtained from a planar mixing configuration, a free jet configuration
and a cantilevered ramp injector configuration.
It is seen that, at a convective Mach number of 1.5 and at a global
equivalence ratio of 1, the mixing efficiency of a cantilevered ramp injector
is as much as 4 times the one of a planar
configuration. This fourfold increase is attributed to, in order of importance:
(i) the increased fuel/air interface length present in 3D, (ii) the higher
pressure and temperature present at injection, and (iii) the stretching of the fuel/air
interface by the axial vortices.


\section{Mixing over a flat plate}

%chapter 6


A parametric study of the effect of the fuel inflow conditions on the mixing
performance of a cantilevered ramp injector  is performed.
It is found that, while keeping the fuel speed and fuel pressure constant, increasing
the global
equivalence ratio from 1 to 3 translates into a 30\% increase in the mixing efficiency,
but results in a large portion
of the mixture at the domain exit to be outside the hydrogen/air flammability limits.
On the other hand, reducing the global equivalence ratio from 1 to 0.33 reduces the mixing
efficiency by 27\%
and induces a high mixture temperature which is beneficial if burning
is desired close to the injection point, but undesirable otherwise. If burning
is not desired near the point of injection, injecting the fuel in stoichiometric
proportions with the incoming air is the recommended approach.
Secondly, keeping the global equivalence ratio and the fuel pressure constant,
the convective Mach number is varied from -0.5 to 1.5, with a negative value indicating
a fuel speed smaller than the air speed. This increase in the convective
Mach number is seen to result in a mixing efficiency increase
of 31\%  while the mass averaged stagnation pressure
decreases by only 10\%. In the near field mixing region,
the convective Mach number has a negligible impact on the mixing efficiency if using
the Wilcox dilatational dissipation. This is
shown to be related to a limitation of the turbulent mixing layer growth
due to compressibility effects occurring at a high turbulent Mach number.
Even when the convective Mach number is 0, a high turbulent Mach number is
present in the near field due to the high local shear stresses
induced by the axial vortices. Due to the presence of a high turbulent Mach number,
the dilatational dissipation reduces considerably the growth of the mixing layer.
The use of the Wilcox dilatational dissipation correction results in a decrease
of the near field mixing efficiency of 25\% and 43\% for a convective
Mach number of 0 and 1.5, respectively.
Thus, one major finding of this thesis is that \emph{the dilatational dissipation
correction affects the mixing efficiency considerably for a cantilevered
ramp injector flowfield, even at a vanishing convective Mach number.}




A parametric study of the variation of the injector array spacing shows that
the mixing efficiency at the domain exit is maximal for an array
spacing equal to the height of the injector. Reducing the spacing diminishes
considerably the axial vortices
strength but increases the interface length at the point
of injection. The rate of growth of the mixing efficiency in the nearfield is observed
to be directly related to the interface length at injection. The
decrease in the axial vortices strength at a small array spacing prevents
enough air from being entrained under the fuel jet, resulting in a
combustible mixture present in the boundary layer.
Due to the higher shock strength present above the injector array, a high angle of injection
translates into significantly more losses: injecting the fuel at 16$^\circ$
results in a twofold increase in the thrust potential losses compared to injection
at 4$^\circ$, with an associated increase in mixing efficiency
of 9\%.

A change in the fuel inflow conditions is observed not to affect the capability of
the cantilevered ramp injector at preventing the injectant from entering the
hot boundary layer. However, a change in the injector geometry
is found to result sometimes in
hydrogen entering the boundary layer. This is attributed to the amount of air separating
the fuel from the wall being strongly dependent on the injector geometry.
The reduction in the air mass flow rate separating the fuel jet from the wall
 is seen to be partly due to weakened
axial vortices and/or to a lower air mass flux flowing under the fuel at the point
of injection. It is observed that an air cushion between the wall and the
hydrogen is sufficiently thick to prevent fuel in the boundary layer
when
(i) the fuel is injected at an angle of approximately $10^\circ$ or more,
(ii) an array spacing of at least the height of the injector is used, and
(iii) a swept ramp configuration is avoided.
Another interesting finding of this thesis is that \emph{the use of a negative sweeping
angle used in conjunction with the cantilevered ramp injector configuration
is observed to result into better fuel penetration and
better mixing, and is highly recommended for mixing in a shcramjet inlet where premature
ignition should be avoided.} Lastly, it is observed that an incoming boundary layer with
a thickness of 15\% the injector height does not diminish significantly the air
cushion between the fuel and the wall.




\section{Mixing in the inlet}
%conclusions from inlet problems

Based on the results obtained from the above-mentioned parametric studies, a baseline
injector configuration is considered as a means to deliver fuel in the inlet
of an external compression shcramjet.
The baseline injector geometry consists of a sweeping angle set to the minimal
value possible, an array spacing equal to the height of the injector, and the
injection angle set to 10 degrees. Two inlet geometries are considered: one in which
the flow is compressed by two equal-strength shocks (\ie\ the shock-shock
configuration), and one in which the second shock is replaced by a Prandtl-Meyer
compression fan (\ie\ the shock-fan configuration). The fuel inflow conditions in
the inlet are such that the
global equivalence ratio approaches 1, and the convective Mach number is 1.2
with the speed of the hydrogen jet higher than the speed of the air.
\emph{Due to the fuel being injected at a very high speed, fuel injection in the inlet
is found to increase considerably the thrust potential, with a
gain exceeding the loss by 40--120\%.}
Another beneficial effect of fuel injection on the inlet performance
is the observed decrease of approximately 10\% in the skin friction force. The decrease
in skin friction is attributed to fuel being present in the boundary
layer after the second inlet compression process: the low density of
hydrogen decreases the density of the boundary layer, hence resulting
in a reduced wall shear stress. However, the presence of fuel in the inlet is not
all beneficial: due to the
Mach number of the hydrogen stream being significantly less than
the Mach number of the air stream, the performance of
the compression fan is reduced, with an associated increase in the
thrust potential losses estimated to be of 8\% for the shock-fan configurations.

\emph{Typically, it is estimated that 50--70\% of the thrust potential
losses in the inlet are due to skin friction.} The large importance of the skin friction
in the shcramjet inlet is seen to be partly due to the axial vortices generated
by the second inlet compression process continuously entraining upwards
the upper part of the boundary layer, which results in a substantial
thinning of the boundary layer, hence increasing the wall shear stress.
The use of a turbulence model that can predict accurately the wall
shear stress, such as the Wilcox $k\omega$ model used herein,
is hence seen to be crucial in assessing the losses accurately in a shcramjet
inlet. Interestingly, no recirculation region is observed near the
second inlet shock for all inlet cases studied. This is not completely
surprising, as turbulent boundary layers are known to offer a
strong resistance to separation as the flow Mach number increases \cite{jfm:1972:coleman}.
Furthermore, both the ramp-generated axial vortices and the axial vortices
generated through the second inlet shock help in reducing the size of the
boundary layer, which reduces the chance of shock-induced flow separation
in the inlet.


The relatively low mixing efficiency of 0.30 obtained for the baseline inlet
cases is attributed to the lack of adequate fuel penetration, partly
due to the absence of a sufficient amount of air separating the fuel
jets upon entering the second inlet compression process. The lack
of air in-between the fuel jets prevents the axial vortices
created by the interaction of the mixing layer with the compression fan (or shock)
to enhance fuel penetration, as they normally do by entraining the air in-between
the fuel jets to under the fuel. \emph{A major difficulty encountered while mixing in
an external compression inlet is that the height of the inlet, and hence
the amount of air entering the inlet, is strongly dependent on the fuel injection process.}
For this reason, contrarily to mixing over a flat plate,
the increase of the injection angle does not translate into a better air-based mixing
efficiency since the increase in the amount of air entering the inlet is more
pronounced than the increase in the amount of air mixed with the fuel.
\emph{One novel approach that is shown
herein to be successful at increasing the fuel penetration
is by alternating the injection angle from
one injector to another.} In this way, the fuel jets emanating from
the high-angle injectors penetrate deeply in the incoming air, while the
fuel jets emanating from the low-angle injectors remain close to the body.
Upon entering the compression process, due to the fuel jets belonging
to two distinct levels, there is a much increased amount of air separating
the fuel jets on each level and the compression process induces better
penetration. Furthermore, the strength of the shock forming on the injector array
does not increase appreciably due to one in every two injector being at a low
angle. The reduced array shock strength  helps in keeping the inlet height
to a low value. The use of alternating injection angles of 9 and 16 degrees
is seen to result in a 32\% increase in the mixing efficiency and a 14\% increase
in the thrust potential losses when compared to injecting the fuel at a single injection
angle of 10 degrees. A second strategy that is shown to increase the fuel penetration
and the mixing efficiency is the use of longer and thinner cantilevered ramp injectors.
By doubling the length of the injector, while doubling the height and halving the depth
of the fuel
jet, a mixing efficiency increase of 15\% is observed at the expense of an increase
in thrust potential losses of 12\%. The combination of the alternating angle
configuration with the longer and thinner injectors results in a mixing efficiency
of 0.47 and 0.44 for the shock-shock inlet configuration and the shock-fan inlet
configuration respectively, which is an increase of more than 50\% over the baseline
injector configuration.


% risk of premature ignition
Premature ignition in the inlet is a high possibility for the baseline inlet
configurations, as a fuel/air mixture is seen to penetrate the
hot boundary layer after the second inlet compression wave. One strategy that is
here shown to prevent to a large extent the fuel
from entering the boundary layer is the use of an increased injector array spacing.
A higher array spacing increases the amount of air that is entrained by the axial vortices
below the fuel jets, hence retarding the complete erosion of the air buffer separating
the fuel from the wall. Unfortunately,
while a higher array spacing prevents the fuel from entering the boundary layer,
it results in stronger axial vortices which entrain upwards the hot flow from the
boundary layer to the mixing layer. This is particularly worrisome as the mixing
layer is then exposed to a temperature significantly above the ignition
point for a considerable amount of time, resulting in a high risk of premature ignition.
Nonetheless, the risk of premature ignition in the mixing layer can be reduced
through the use of a shock-fan configuration, which results in
a reduction in the flow temperature of as much as 80~K compared to the
shock-shock configuration. Besides helping to prevent the risk of premature ignition,
a reduction of the temperature of the flow entering the combustor augments the
performance of the shcramjet due to a
more efficient heat released in the shock-induced combustion \cite{jpp:2001:sislian}.
\emph{The use of a Prandtl-Meyer compression surface in a shcramjet inlet is hence strongly
recommended as it decreases the thrust potential losses, reduces
the risk of premature ignition, while resulting in a small 6\% diminution of the
mixing efficiency for the optimal injector configuration considered.}


