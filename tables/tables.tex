\documentclass[oneside,onecolumn,a4paper,10pt,notitlepage]{tables}
% can use option mtpro 

\usepackage{graphicx}
\usepackage{rotating}
\usepackage{scalefnt}

\renewcommand\arraystretch{2.0}

\newcommand{\mfd}{\displaystyle}



\begin{document}

\pagestyle{empty}

\begin{center}
\Huge \bf{Viscous Fluid Flow Tables}
\end{center}
\vfill
\vfill
Mass Conservation Equation in Cartesian Coordinates:\\[0.5em]
$\mfd \frac{\partial \rho}{\partial t} + \frac{\partial }{\partial x}(\rho u) +  \frac{\partial }{\partial y}(\rho v) +  \frac{\partial }{\partial z}(\rho w) 
=
0
$\\
\vfill
Mass Conservation Equations in Cylindrical Coordinates:\\[0.3em]
$\mfd
\frac{\partial \rho}{\partial t} + \frac{1}{r} \frac{\partial }{\partial r} (\rho r v_r) + \frac{1}{r} \frac{\partial }{\partial \theta}(\rho v_\theta) +  \frac{\partial}{\partial x}(\rho v_x)=0
$\\

\vfill
Navier-Stokes Equations in Cartesian Coordinates:\\[0.3em]
$
\mfd \frac{\partial \rho u}{\partial t} + \frac{\partial \rho u^2}{\partial x} + \frac{\partial \rho u v}{\partial y} +  \frac{\partial \rho u w}{\partial z}
=
-\frac{\partial P}{\partial x}
+
 \frac{\partial \tau_{xx}}{\partial x}  
+
 \frac{\partial \tau_{yx}}{\partial y}  
+
 \frac{\partial \tau_{zx}}{\partial z}  
+B_x 
$\\[0.3em]
$
\mfd \frac{\partial \rho v}{\partial t} + \frac{\partial \rho uv}{\partial x} + \frac{\partial \rho v^2}{\partial y} +  \frac{\partial \rho wv}{\partial z}
=
-\frac{\partial P}{\partial y}
+
 \frac{\partial \tau_{xy}}{\partial x}  
+
 \frac{\partial \tau_{yy}}{\partial y}  
+
 \frac{\partial \tau_{zy}}{\partial z}  
+B_y
$\\[0.3em]
$
\mfd \frac{\partial \rho w}{\partial t} + \frac{\partial \rho uw}{\partial x} + \frac{\partial \rho vw}{\partial y} +  \frac{\partial \rho w^2}{\partial z}
=
-\frac{\partial P}{\partial z}
+
 \frac{\partial \tau_{xz}}{\partial x}  
+
 \frac{\partial \tau_{yz}}{\partial y}  
+
 \frac{\partial \tau_{zz}}{\partial z}  
+B_z
$\\[0.3em]
\vfill
with the shear stresses equal to:\\[0.3em]
%
$
\tau_{xx}=\mu\left(\frac{4}{3}  \frac{\partial u}{\partial x}-\frac{2}{3}\frac{\partial u}{\partial y}-\frac{2}{3}\frac{\partial u}{\partial z}\right)~~~~
\tau_{yy}=\mu\left(\frac{4}{3}  \frac{\partial v}{\partial y}-\frac{2}{3}\frac{\partial v}{\partial x}-\frac{2}{3}\frac{\partial v}{\partial z}\right)~~~~
\tau_{zz}=\mu\left(\frac{4}{3}  \frac{\partial w}{\partial z}-\frac{2}{3}\frac{\partial w}{\partial x}-\frac{2}{3}\frac{\partial w}{\partial y}\right)
$\\[0.3em]
$
\tau_{xy}=\tau_{yx}=\mu\left(\frac{\partial u}{\partial y}+\frac{\partial v}{\partial x} \right)~~~~
\tau_{xz}=\tau_{zx}=\mu\left(\frac{\partial u}{\partial z}+\frac{\partial w}{\partial x} \right)~~~~
\tau_{yz}=\tau_{zy}=\mu\left(\frac{\partial v}{\partial z}+\frac{\partial w}{\partial y} \right)
$\\[0.3em]
%
\vfill
Navier-Stokes Equations in Cartesian Coordinates (constant $\rho$, constant $\mu$):\\[0.3em]
$\mfd \rho \left(\frac{\partial u}{\partial t} + u \frac{\partial u}{\partial x} + v \frac{\partial u}{\partial y} + w \frac{\partial u}{\partial z} \right)
=
-\frac{\partial P}{\partial x}
+
\mu  \frac{\partial^2 u}{\partial x^2}  
+
\mu  \frac{\partial^2 u}{\partial y^2}  
+
\mu  \frac{\partial^2 u}{\partial z^2}  +B_x
$\\[0.3em]
%
$\mfd \rho \left(\frac{\partial v}{\partial t} + u \frac{\partial v}{\partial x} + v \frac{\partial v}{\partial y} + w \frac{\partial v}{\partial z} \right)
=
-\frac{\partial P}{\partial y}
+
\mu  \frac{\partial^2 v}{\partial x^2}  
+
\mu  \frac{\partial^2 v}{\partial y^2}  
+
\mu  \frac{\partial^2 v}{\partial z^2}  +B_y
$\\[0.3em]
%
$\mfd \rho \left(\frac{\partial w}{\partial t} + u \frac{\partial w}{\partial x} + v \frac{\partial w}{\partial y} + w \frac{\partial w}{\partial z} \right)
=
-\frac{\partial P}{\partial z}
+ 
\mu  \frac{\partial^2 w}{\partial x^2}  
+
\mu  \frac{\partial^2 w}{\partial y^2}  
+
\mu  \frac{\partial^2 w}{\partial z^2}  +B_z
$\\[0.8em]
\vfill
Navier-Stokes Equations in Axisymmetric Coordinates (constant $\rho$, constant $\mu$):\\[0.3em]
$
\mfd \rho \left(\frac{\partial v_x}{\partial t}+ v_x \frac{\partial v_x}{\partial x} + v_r \frac{\partial v_x}{\partial r} \right)=-\frac{\partial P}{\partial x}+ \mu\frac{\partial^2 v_x}{\partial x^2}+\frac{\mu}{r} \frac{\partial }{\partial r}\left(r \frac{\partial v_x}{\partial r} \right) +B_x
$\\[0.3em]
$
\mfd \rho \left(\frac{\partial v_r}{\partial t}+ v_x \frac{\partial v_r}{\partial x} + v_r \frac{\partial v_r}{\partial r} \right)=-\frac{\partial P}{\partial r}+ \mu\frac{\partial^2 v_r}{\partial x^2}+\frac{\mu}{r} \frac{\partial }{\partial r}\left(r \frac{\partial v_r}{\partial r} \right) -\mu\frac{v_r}{r^2} +B_r
$\\[0.8em]
\vfill
Navier-Stokes Equations in Cylindrical Coordinates (constant $\rho$, constant $\mu$):\\[0.3em]
$
\mfd\rho\left(\frac{\partial v_r}{\partial t} + v_r \frac{\partial v_r}{\partial r} + \frac{v_\theta}{r} \frac{\partial v_r}{\partial \theta} - \frac{v_\theta^2}{r} + v_x \frac{\partial v_r}{\partial x} \right) = B_r - \frac{\partial P}{\partial r} 
+ \frac{\mu}{r}\frac{\partial}{\partial r}\left(r \frac{\partial v_r}{\partial r}\right)
   + \frac{\mu}{r^2} \frac{\partial^2 v_r}{\partial \theta^2} - \frac{2\mu}{r^2}\frac{\partial v_\theta}{\partial \theta}
-\mu \frac{v_r}{r^2}
+ \mu \frac{\partial^2 v_r}{\partial x^2} 
$\\[0.3em]
$
\mfd\rho\left(\frac{\partial v_\theta}{\partial t} + v_r \frac{\partial v_\theta}{\partial r} + \frac{v_\theta}{r} \frac{\partial v_\theta}{\partial \theta} + \frac{v_r v_\theta}{r} + v_x \frac{\partial v_\theta}{\partial x} \right) 
= B_\theta - \frac{1}{r}\frac{\partial P}{\partial \theta} 
+ \frac{\mu}{r}\frac{\partial}{\partial r}\left(r \frac{\partial v_\theta}{\partial r} \right)
   + \frac{\mu}{r^2} \frac{\partial^2 v_\theta}{\partial \theta^2} + \frac{2\mu}{r^2}\frac{\partial v_r}{\partial \theta}+ \mu \frac{\partial^2 v_\theta}{\partial x^2}
- \mu \frac{v_\theta}{r^2}
$\\
$
\mfd\rho\left(\frac{\partial v_x}{\partial t} + v_r \frac{\partial v_x}{\partial r} + \frac{v_\theta}{r} \frac{\partial v_x}{\partial \theta} + v_x \frac{\partial v_x}{\partial x} \right) = B_x - \frac{\partial P}{\partial x} + \frac{\mu}{r}\frac{\partial}{\partial r}\left(r \frac{\partial v_x}{\partial r} \right)
   + \frac{\mu}{r^2} \frac{\partial^2 v_x}{\partial \theta^2} + \mu \frac{\partial^2 v_x}{\partial x^2}
$\\
~\\
\vfill



\newpage


\vfill
First law of thermodynamics in integral form:\\[0.5em]
$\mfd \frac{{\rm d}}{{\rm d}t} \int_{V} \rho E {\rm d}V
+ \int_{S} (\rho E + P)(\vec{v}\cdot\vec{n}){\rm d}S=\dot{Q}-\dot{W}$\\[0.5em]
with $E$ the total energy (i.e. $E\equiv e + q^2/2 + g H$) where $e$ is the internal energy of the fluid and $H$ the height from the ground.\\ 

\vfill
Energy equation in Cartesian coordinates:\\[0.5em]
$
\mfd \frac{\partial \rho E}{\partial t} + \frac{\partial \rho u H}{\partial x} + \frac{\partial \rho v H}{\partial y} + \frac{\partial \rho w H}{\partial z} 
=
 \frac{\partial}{\partial x}\left( k \frac{\partial T}{\partial x} \right) 
+\frac{\partial}{\partial y}\left( k \frac{\partial T}{\partial y} \right) 
+\frac{\partial}{\partial z}\left(k \frac{\partial T}{\partial z} \right) 
$
\\[0.3em]
$
~~~~~~~
\mfd 
+\frac{\partial u \tau_{xx}}{\partial x}
+\frac{\partial u \tau_{yx}}{\partial y}
+\frac{\partial u \tau_{zx}}{\partial z}
+\frac{\partial v \tau_{xy}}{\partial x}
+\frac{\partial v \tau_{zy}}{\partial z}
+\frac{\partial v \tau_{yy}}{\partial y}
+\frac{\partial w \tau_{xz}}{\partial x}
+\frac{\partial w \tau_{yz}}{\partial y}
+\frac{\partial w \tau_{zz}}{\partial z}
$\\[0.8em]
\vfill
Energy equation in Cartesian coordinates (constant $\rho$, constant $\mu$):\\[0.3em]
$
\mfd \rho \left(\frac{\partial e}{\partial t} + u \frac{\partial e}{\partial x} + v \frac{\partial e}{\partial y} + w \frac{\partial e}{\partial z} \right)
=
 \frac{\partial}{\partial x}\left(k \frac{\partial T}{\partial x} \right) + \frac{\partial}{\partial y}\left(k \frac{\partial T}{\partial y} \right) +\frac{\partial}{\partial z}\left(k \frac{\partial T}{\partial z} \right)
+       \mu \left(\frac{\partial u}{\partial x} \right)^2 
	+ \mu \left(\frac{\partial u}{\partial y} \right)^2
	+ \mu \left(\frac{\partial u}{\partial z} \right)^2
$ \\[0.9em]
~~~~~~~$\mfd
	+ \mu \left(\frac{\partial v}{\partial x} \right)^2 
	+ \mu \left(\frac{\partial v}{\partial y} \right)^2
	+ \mu \left(\frac{\partial v}{\partial z} \right)^2
	+ \mu \left(\frac{\partial w}{\partial x} \right)^2 
	+ \mu \left(\frac{\partial w}{\partial y} \right)^2
	+ \mu \left(\frac{\partial w}{\partial z} \right)^2
$\\[0.8em]
\vfill
Energy equation in Axisymmetric coordinates (constant $\rho$, constant $\mu$):\\[0.3em]
$
\mfd \rho \left(\frac{\partial e}{\partial t} + u \frac{\partial e}{\partial x} + v \frac{\partial e}{\partial r}  \right)
=
  \frac{\partial }{\partial x}\left(k\frac{\partial T}{\partial x} \right) + \frac{1}{r} \frac{\partial }{\partial r} \left(kr \frac{\partial T}{\partial r} \right) 
+       \mu \left(\frac{\partial u}{\partial x} \right)^2 
	+ \mu \left(\frac{\partial u}{\partial r} \right)^2
	+ \mu \left(\frac{\partial v}{\partial x} \right)^2 
	+ \mu \left(\frac{\partial v}{\partial r} \right)^2
$\\[0.8em]
In the latter, $e$ is the internal energy of the fluid (which can be taken equal to $c T$ for a calorically perfect gas or liquid), $\tau$ is the shear stress, $E$ the total energy equal to $e+\frac{1}{2} U^2$, $H$ the total enthalpy equal to $h+\frac{1}{2} U^2$, and $h$ the enthalpy equal to $e+P/\rho$. For a liquid, the heat capacity at constant volume $c$ is essentially the same as the heat capacity at constant pressure $c_p$.\\[1.0em]
\vfill

Conservation of energy in pipes (constant $\rho$, uniform $P$, $e$, and $H$ at entrance and exit):\\[0.5em]
$\mfd \left( \frac{P_1}{\rho} + g y_1 + \frac{\alpha_1}{2} (u_{\rm b})_1^2\right)
-\left( \frac{P_2}{\rho} + g y_2 + \frac{\alpha_2}{2} (u_{\rm b})_2^2\right)=h_{\rm L}$\\[0.5em]
with $g$ the gravitational acceleration, $y$ the height from the ground, $\alpha$ the kinetic energy coefficient, and $h_{\rm L}$ the head loss. \\
~\\

\vfill



Reynolds number: \\[0.4em]
$\mfd {\rm Re}_L \equiv  \frac{\rho U L}{\mu} = \frac{{\rm rate~of~convection}~(\rho U)}{{\rm rate~of~viscous~diffusion}~(\mu/L)}$\\[0.5em]
where $\rho$ is the density of the fluid in $\rm kg/m^3$, $U$ is the speed of the fluid in m/s, $L$ is a characterictic length in meters, and $\mu$ is the viscosity of the fluid in kg/ms. The flow becomes turbulent when  ${\rm Re}_L \gtrsim 5 \times 10^5$ for a flat plate, and when ${\rm Re}_D \gtrsim 2300$ for a pipe.\\
\vfill
Skin friction coefficient: \\[0.4em]
$\mfd C_f \equiv  \frac{\tau_{\rm w}}{\frac{1}{2} \rho q_\infty^2}$ \\[0.5em]
where $\tau_{\rm w}$ is the shear stress at the wall.\\
\vfill
Friction factor in pipes (Darcy friction factor): \\[0.4em]
$\mfd f \equiv  \frac{4 \tau_{\rm w}}{\frac{1}{2} \rho u_{\rm b}^2}$ \\[0.5em]
where $\tau_{\rm w}$ is the shear stress at the wall.\\
\vfill
Hydraulic Diameter:\\[0.5em]
$\mfd D_{\rm H}=4 A_{\rm cs} / \xi$\\[0.5em]
with $A_{\rm cs}$ the cross-section and $\xi$ the perimeter of the duct.\\
\vfill

\newpage




Boundary layer thickness for flow over flat plates:\\[0.5em]
\begin{tabular*}{\textwidth}{lll}
\toprule
\textbf{\em Flow regime} & \textbf{\em Restrictions / Comments} & \textbf{\em Equation} \\
\midrule
\begin{minipage}{1.1in}
  Laminar 
\end{minipage} 
  & \begin{minipage}{3.2in}
     ${\rm Re}_x< 5\cdot 10^5$
    \end{minipage}
  & $\mfd {\delta}/{x}=4.64 \cdot {\rm Re}_x^{-1/2}$\\
\begin{minipage}{1.1in}
  Turbulent 
\end{minipage} 
  & \begin{minipage}{3.2in}
     ${\rm Re}_x< 10^7$~,~~ $\delta=0$~ at ~$x=0$
    \end{minipage}
  & $\mfd {\delta}/{x}=0.381 \cdot {\rm Re}_x^{-1/5}$\\
\begin{minipage}{1.1in}
  Turbulent 
\end{minipage} 
  & \begin{minipage}{3.2in}
     ${\rm Re_{crit}}<{\rm Re}_x< 10^7$~,~~ ${\rm Re_{crit}}=5 \cdot 10^5$~,~~$\delta=\delta_{\rm lam}$~ at ~$\rm Re_{crit}$ 
    \end{minipage}
  & $\mfd {\delta}/{x}=0.381 \cdot {\rm Re}_x^{-1/5}-10256 \cdot {\rm Re}_x^{-1}$\\
\bottomrule
\end{tabular*}
~\\
\vfill




Friction coefficient for flow over flat plates:\\[0.5em]
\begin{tabular*}{\textwidth}{lll}
\toprule
\textbf{\em Flow regime} & \textbf{\em Restrictions / Comments} & \textbf{\em Equation} \\
\midrule
  Laminar,~local
  & ${\rm Re}_x< 5\cdot 10^5$
  & $\mfd C_{f}=0.647 \cdot {\rm Re}_x^{-1/2}$\\
  Turbulent,~local
  & $5\cdot 10^5 < {\rm Re}_x< 10^7$
  & $\mfd C_{f}=0.0592 \cdot {\rm Re}_x^{-1/5}$\\
  Turbulent,~local
  & $10^7 < {\rm Re}_x< 10^9$
  & $\mfd C_{f}=0.37 \cdot ({\rm ln}\,\, {\rm Re}_x)^{-2.584}$\\
  Turbulent,~average
  & ${\rm Re_{crit}} < {\rm Re}_x< 10^9$
  & $\mfd \overline{C_{f}}=0.455 \cdot ({\rm ln}\,\, {\rm Re}_L)^{-2.584}-A \cdot {\rm Re}_L^{-1}$\\
  & \begin{minipage}{3.2in}
    ~\\[0.2em]
      \begin{tabular}{@{}l@{~~~~:~~~~}llll@{}}
       
       ${\rm Re_{crit}}$ & $3\cdot 10^5$ & $5 \cdot 10^5$ & $10^6$  & $3\cdot 10^6$ \\
       
       $A$             & 1055          & 1742           & 3340    & 8940 \\       
       
     \end{tabular}
    \end{minipage}
  & ~\\
\bottomrule
\end{tabular*}\\[0.4em]
~\\
\vfill


Friction factor for internal flow in a circular tube of length $L$ and diameter $D$:\\[0.5em]
\begin{tabular*}{\textwidth}{lll}
\toprule
\textbf{\em Flow regime} & \textbf{\em Restrictions / Comments} & \textbf{\em Equation} \\
\midrule
  Fully-developed~laminar~flow
  & 
     Range of applicability: ${\rm Re}_D \lesssim 2300$. 
  & $\mfd f={64}/{{\rm Re}_D}$\\
\begin{minipage}{1.76in}
  Fully-developed turbulent flow\\(smooth tubes)
\end{minipage} 
  & \begin{minipage}{2.92in}
     Range of applicability: $2300 \lesssim {\rm Re}_D \lesssim 2 \times 10^4$.\\[0.3em]
    \end{minipage}
  & $\mfd f=0.316 \, {\rm Re}_D^{-1/4}$\\
\begin{minipage}{1.76in}
  Fully-developed turbulent flow\\(smooth tubes)
\end{minipage} 
  & \begin{minipage}{2.92in}
     Range of applicability: ${\rm Re}_D \gtrsim 2 \times 10^4$.\\[0.3em]
    \end{minipage}
  & $\mfd f=0.184 \, {\rm Re}_D^{-1/5}$\\
\begin{minipage}{1.76in}
  Fully-developed turbulent flow\\(smooth tubes)
\end{minipage} 
  & \begin{minipage}{2.92in}
     Range of applicability: $3000 \lesssim {\rm Re}_D \lesssim 5 \times 10^6$.\\[0.3em]
    \end{minipage}
  & $\mfd f=\left( 0.79 \, \ln ({\rm Re}_D) -1.64\right)^{-2}$\\
\bottomrule
\end{tabular*}\\[0.4em]
For flow in ducts with a non-circular cross section, the above correlations can still be used provided the diameter $D$ is substituted by the hydraulic diameter\\
\vfill


Fluid Properties at 1 atm and 27$^\circ$C:\\[0.5em]
\begin{tabular*}{\textwidth}{lcc}
\toprule
Fluid & density $\rho$, kg/m$^3$ & viscosity $\mu$, kg/ms \\
\midrule
Oxygen & 1.30 & $2.06 \cdot 10^{-5}$\\
Nitrogen & 1.14 & $1.78 \cdot 10^{-5}$\\
Air   & 1.20  & $1.84 \cdot 10^{-5}$ \\
Water & 995.8 & $8.60 \cdot 10^{-4}$\\
\bottomrule
\end{tabular*}
~\\
~\\
\vfill
\newpage

\vfill
Standard value of the pressure in the atmosphere at sea-level:\\
$
\mfd
P_{\rm atm}=\left\{ 
\begin{array}{lll}
1 & {\rm atm} & \textrm{(atmospheres)}\\
759.81 & {\rm torr} & \textrm{(Torr)}\\
14.7 & {\rm psi} & \textrm{(pounds per square inch)}\\
101 & {\rm kPa} & \textrm{(kiloPascals)}\\
760 & {\rm mmHg} & \textrm{(millimeters of mercury)}\\
2116.8 & {\rm psf} & \textrm{(pounds per square foot)}
\end{array}
 \right.
$
\vfill\vfill\vfill
\vfill\vfill\vfill
\vfill\vfill\vfill
\vfill\vfill\vfill
\vfill\vfill\vfill
~\\
\newpage
\includegraphics[width=\textwidth]{moody.pdf}



\end{document}


%%% new table
Title:\\[0.5em]
\begin{tabular*}{\textwidth}{lcc}
\toprule
Col1 title & Col2 title & Col3 title \\
\midrule
col1 & col2 & col3\\
\bottomrule
\end{tabular*}


%%% new figure
\includegraphics[width=\textwidth]{fig.pdf}


%%% changing the array stretch 
\renewcommand\arraystretch{1.3}


